\subsection{Textury postaviček}
Postavičky (to jest já nebo nepřítel) mají za texturu sérii obrázků, jsou tedy animované (nyní je jedno, že jich může být několik druhů v závislosti na tom, zdali jsem například ve vzduchu nebo na zemi). Na to slouží třída \t{AnimatedSprite} a \t{AnimatedSpriteHead}. Rozdíl mezi nimi je ten, že "já" mám ve hře hlavu a tělo animované zvlášť (z několika důvodů, například aby bylo snažší měnit hlavu v závislosti na levelu, nebo aby hlava a tělo mohly mít jiný počet snímků). Nyní budu popisovat jen \t{AnimatedSprite}, druhá třída je zcela analogická s přidaným bonusem hlavy, která je tam zvlášť (je tam navíc ještě vrstva s rukou, jelikož třeba za letu musí být ruka nad hlavou a hlava nad tělem).
\subsubsection{AnimatedSprite}
Konstruktor této třídy dostane texturu, což je tabulka \t{rows}$\times$\t{columns} obrázků (její rozměry konstruktor dostane taky). \\
Z funkcí stojí za zmínku \t{Update}, která přejde k dalšímu snímku, \t{stop}, která se vrátí na začátek posloupnosti snímků (když se zastavím, tak mám stát a ne být v půlce kroku)\footnote{Ve třídě \t{AnimatedSpriteHead} se vrátí na začátek pouze tělo, hlava se pohybuje dál i když panáček stojí.}, a dále klasická dvojice funkcí \t{Update} a \t{Draw}. V \t{Update} se pouze přejde k dalšímu obrázku v posloupnosti (a je na tom, kdo vyrobil instanci této třídy aby se rozhodl kdy to chce dělat). A v \t{Draw} se vybere z \t{Texture} správný obdélník který se vykreslí na dané souřadnice a danou velikostí (dané jako parametry funkce \t{draw}).
